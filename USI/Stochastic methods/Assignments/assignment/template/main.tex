\documentclass[unicode,11pt,a4paper,oneside,numbers=endperiod,openany]{scrartcl}

\usepackage{assignment}
\usepackage{textcomp}
\usepackage{amsmath}
\usepackage{theorem}
\usepackage{algorithm,algorithmic}
\hyphenation{PageRank}
\hyphenation{PageRanks}

\newtheorem{theorem}{Theorem}
\renewcommand\thesubsection{\arabic{subsection}}
\DeclareOldFontCommand{\bf}{\normalfont\bfseries}{\mathbf}
\DeclareOldFontCommand{\it}{\normalfont\bfseries}{\mathit}


\allowdisplaybreaks

\begin{document}

\setassignment
\setduedate{Friday, May 14, 2021, 11.59pm}
\serieheader{Stochastic Methods}{Academic Year 2020/2021}{Prof. Dr. Illia Horenko (illia.horenko@usi.ch)}{Edoardo Vecchi (edoardo.vecchi@usi.ch)}{Assignment 4 - Solution}{[Ashutosh Singh]}

%-----------------------------------------------------------------------------------------------

\section*{Exercise 1: Inconsistent Systems of Equations}

\begin{itemize}
	\item [(a)]
	\begin{equation}
	    A_1x = b_1
	\end{equation}
	\[
        \begin{bmatrix}
        1 & 0 \\
        1 & 0 \\
        1 & 0 \\
        \end{bmatrix}
        \begin{bmatrix}
        x_1 \\ x_2 \\ 
        \end{bmatrix}
        =
        \begin{bmatrix}
        5 \\ 2 \\ 4
        \end{bmatrix}
    \]
    
    \begin{equation}
	    A_1^TA_1x = A_1^Tb_1
	\end{equation}
    \[
    	\begin{bmatrix}
            1 & 1 & 1 \\
            0 & 0 & 0\\
        \end{bmatrix}
        \begin{bmatrix}
            1 & 0 \\
            1 & 0 \\
            1 & 0 \\
        \end{bmatrix}
        \begin{bmatrix}
        x_1 \\ x_2 \\ 
        \end{bmatrix}
        =
        \begin{bmatrix}
            1 & 1 & 1 \\
            0 & 0 & 0\\
        \end{bmatrix}
        \begin{bmatrix}
        5 \\ 2 \\ 4
        \end{bmatrix}
    \]
    \[
        \begin{bmatrix}
        3 & 0 \\
        0 & 0 \\
        \end{bmatrix}
        \begin{bmatrix}
        x_1 \\ x_2 \\ 
        \end{bmatrix}
        =
        \begin{bmatrix}
        11 \\ 0
        \end{bmatrix}
    \]
    \begin{equation}
	    \Rightarrow x_1^* = \frac{11}{3};
	    \\
	    \Rightarrow x_2^* = k
	\end{equation}
    
	{Residual}\\
	 \begin{equation}
	    r = b_1 - A_1x^*
	 \end{equation}
	 
	 \[
	    r = 
	    \begin{bmatrix}
        5 \\ 2 \\ 4
        \end{bmatrix}
        -
        \begin{bmatrix}
        \frac{11}{3} \\
        \frac{11}{3} \\
        \frac{11}{3} \\
        \end{bmatrix}
    \]
    \[
	    r = 
        \begin{bmatrix}
        \frac{4}{3} \\
        -\frac{5}{3} \\
        \frac{1}{3} \\
        \end{bmatrix}
    \]
    
    {Euclidean norm of Residual}\\
     \begin{equation}
	    ||r||_2 = \sqrt{(\frac{4}{3})^2 + (-\frac{5}{3})^2 + (\frac{1}{3})^2}
	 \end{equation}
	 \begin{equation}
	    ||r||_2 = 2.16
	 \end{equation}
	 
    
    {SE of Residual}\\
    \begin{equation}
	    SE = ||r||_2^2
	 \end{equation}
	 \begin{equation}
	    SE = 4.67
	 \end{equation}
    
    {RMSE of Residual}\\
    \begin{equation}
	    RMSE = \sqrt{(\frac{SE}{m})}
	 \end{equation}
	 \begin{equation}
	    RMSE = \sqrt{(\frac{4.67}{3})} = 1.247
	 \end{equation}
	
	
	\item [(b)]
	
	\begin{equation}
	    A_2x = b_2
	\end{equation}
	\[
        \begin{bmatrix}
        1 & 0 & 0 \\
        0 & 1 & 1 \\
        1 & 2 & 1 \\
        1 & 0 & 1 \\
        \end{bmatrix}
        \begin{bmatrix}
        x_1 \\ x_2 \\ x_3 \\
        \end{bmatrix}
        =
        \begin{bmatrix}
        2 \\ 2 \\ 3 \\ 4
        \end{bmatrix}
    \]
    
    \begin{equation}
	    A_2^TA_2x = A_2^Tb_2
	\end{equation}
    \[
    	\begin{bmatrix}
            1 & 0 & 1 & 1 \\
            0 & 1 & 2 & 0 \\
            0 & 1 & 1 & 1 \\
        \end{bmatrix}
        \begin{bmatrix}
           1 & 0 & 0 \\
            0 & 1 & 1 \\
            1 & 2 & 1 \\
            1 & 0 & 1 \\
        \end{bmatrix}
        \begin{bmatrix}
            x_1 \\ x_2 \\ x_3 \\
        \end{bmatrix}
        =
        \begin{bmatrix}
            1 & 0 & 1 & 1 \\
            0 & 1 & 2 & 0 \\
            0 & 1 & 1 & 1 \\
        \end{bmatrix}
        \begin{bmatrix}
            2 \\ 2 \\ 3 \\ 4
        \end{bmatrix}
    \]
    \[
        \begin{bmatrix}
        3 & 3 & 2\\
        3 & 6 & 3 \\
        2 & 3 & 3 \\
        \end{bmatrix}
        \begin{bmatrix}
        x_1 \\ x_2 \\ x_3 \\
        \end{bmatrix}
        =
        \begin{bmatrix}
        9 \\ 10 \\ 9
        \end{bmatrix}
    \]
    
    {After gaussian elimination}
    \[
        \begin{bmatrix}
        3 & 3 & 2\\
        0 & 3 & 1 \\
        0 & 0 & \frac{4}{3} \\
        \end{bmatrix}
        \begin{bmatrix}
        x_1^* \\ x_2^* \\ x_3^* \\
        \end{bmatrix}
        =
        \begin{bmatrix}
        9 \\ 1 \\ \frac{8}{3}
        \end{bmatrix}
    \]
    \begin{equation}
	    \Rightarrow x_1^* = 2;
	    \\
	    \Rightarrow x_2^* = -\frac{1}{3};
	    \\
	    \Rightarrow x_3^* = 2
	\end{equation}
	
	
	{Residual}\\
	 \begin{equation}
	    r = b_2 - A_2x^*
	 \end{equation}
	 
	 \[
	    r = 
	    \begin{bmatrix}
        2 \\ 2 \\ 3 \\ 4
        \end{bmatrix}
        -
        \begin{bmatrix}
        \frac{5}{3} \\
        \frac{5}{3} \\
        \frac{10}{3} \\
        4 \\
        \end{bmatrix}
    \]
    \[
	    r = 
        \begin{bmatrix}
        \frac{1}{3} \\
        \frac{1}{3} \\
        -\frac{1}{3} \\
        0
        \end{bmatrix}
    \]
    
    {Euclidean norm of Residual}\\
     \begin{equation}
	    ||r||_2 = \sqrt{(\frac{1}{3})^2 + (\frac{1}{3})^2 + (-\frac{1}{3})^2 + 0}
	 \end{equation}
	 \begin{equation}
	    ||r||_2 = 0.577
	 \end{equation}
	 
    
    {SE of Residual}\\
    \begin{equation}
	    SE = ||r||_2^2
	 \end{equation}
	 \begin{equation}
	    SE = \frac{1}{3}
	 \end{equation}
    
    {RMSE of Residual}\\
    \begin{equation}
	    RMSE = \sqrt{(\frac{SE}{m})}
	 \end{equation}
	 \begin{equation}
	    RMSE = \sqrt{(\frac{\frac{1}{3}}{4})} = \sqrt{\frac{1}{12}}
	 \end{equation}
\end{itemize}

%------------------------------------------------------------------------------------

\section*{Exercise 2: Comparison of Polynomials Models for Least Squares}

\begin{itemize}
	\item [(a)] 
	\item [(b)] 
	\item [(c)] 
	\item [(d)] 
	\item [(e)]
\end{itemize}

%-----------------------------------------------------------------------------------------

\section*{Exercise 3: Analysis of Periodic Data}

\begin{itemize}
	\item [(a)] 
	\item [(b)] 
	\item [(c)]
\end{itemize}

%-----------------------------------------------------------------------------------------

\section*{Exercise 4: Linearization and Levenberg-Marquardt Method for Exponential Model}

\begin{itemize}
	\item [(a)] 
	\item [(b)] 
	\item [(c)]
\end{itemize}


%-----------------------------------------------------------------------------------------

\section*{Exercise 5: Tikhonov Regularization}

\begin{itemize}
	\item [(a)] 
	\item [(b)] 
	\item [(c)]
\end{itemize}


%-----------------------------------------------------------------------------------------


\end{document}
