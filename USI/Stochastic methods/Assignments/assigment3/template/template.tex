\documentclass[unicode,11pt,a4paper,oneside,numbers=endperiod,openany]{scrartcl}

\usepackage{assignment}
\usepackage{textcomp}
\usepackage{amsmath}
\usepackage{theorem}
\usepackage{algorithm,algorithmic}
\hyphenation{PageRank}
\hyphenation{PageRanks}

\newtheorem{theorem}{Theorem}
\renewcommand\thesubsection{\arabic{subsection}}
\DeclareOldFontCommand{\bf}{\normalfont\bfseries}{\mathbf}
\DeclareOldFontCommand{\it}{\normalfont\bfseries}{\mathit}
% \DeclarePairedDelimiter\abs{\lvert}{\rvert}%


\allowdisplaybreaks

\begin{document}

\setassignment
\setduedate{Tuesday, April 27, 2021, 11.59pm}
\serieheader{Stochastic Methods}{Academic Year 2020/2021}{Prof. Dr. Illia Horenko (illia.horenko@usi.ch)}{Edoardo Vecchi (edoardo.vecchi@usi.ch)}{Assignment 3 - Solution}{[Ashutosh Singh]}

%-----------------------------------------------------------------------------------------------

\section*{Exercise 1: Unconstrained Optimization of a Quadratic Problem}

{Solution}

${f(x) = x^THx + c^Tx}$

${f'(x) = 2Hx + c}$

{For extrema of the function. Set first derivative equal to zero}

${f'(x) = 0}$

${2Hx + c = 0}$

\Rightarrow {x = \frac{-(H^{-1}c)}{2}}

\section*{Exercise 2: Constrained Optimization with Parametrization}
{Solution}



{\max_{x_1, x_2} f(x_1,x_2) = 5 − {{x_1}^{2}} -  { \frac{1}{2} {x_2}^{2}}}

{s.t. x_1+x_2=2}\\~\\

{Converting it to minimization problem by setting g(x) = -f(x)}

{\min_{x_1, x_2} g(x_1,x_2) = - 5 + {{x_1}^{2}} + { \frac{1}{2} {x_2}^{2}}}

{s.t. x_1+x_2=2}\\~\\

{Now for parametrization, Put $x_1 = t$}

\Rightarrow{ x_2 = 2 - t}

\Rightarrow{ g(x_1,x_2) = h(t) = - 5 + t^2 + \frac{(t - 2)^2}{2}}

\Rightarrow{ h(t) = - 3 + \frac{3{t^2}}{2} - 2t}\\~\\

{Now our minimization problem becomes}

{\min_{t} h(t) = - 3 + \frac{3{t^2}}{2} - 2t}

\Rightarrow{h'(t) = 3t - 2 = 0}

\Rightarrow{t = \frac{2}{3}}

\Rightarrow{x_1 = \frac{2}{3}}

\Rightarrow{x_2 = 2 - \frac{2}{3} = 4/3}




\section*{Exercise 3: Optimization on the Unit Circle}
\begin{enumerate}
	\item [(a)] 
	
	All norm balls are in the figure.
	\item [(b)] 
	
	Solution
	
	{$f(x_1,x_2) = x_1x_2$}
	
	{For 1-norm unit ball constraint in first quadrant}
	
	{\min_{x_1, x_2} f(x_1,x_2) }

    {s.t. |x_1|+ |x_2| = 1}\\~\\
    
    {Writing in Lagrange form}
    
    {$L(x_1, x_2, \lambda) = x_1x_2 + λ(|x_1|+ |x_2| - 1)$}
    
    {\nabla L(x_1, x_2, \lambda) = \left[\begin{array}{c}
\dfrac{\partial f}{\partial x_1}(\left.x_{1}, x_{2}, \lambda \right)\\
\dfrac{\partial f}{\partial x_2}(\left.x_{1}, x_{2}, \lambda \right) \\
\dfrac{\partial f}{\partial \lambda}(\left.x_{1}, x_{2}, \lambda }\right)
\end{array}\right]}

    {\nabla L(x_1, x_2, \lambda) = \left[\begin{array}{c}
x_2 + \lambda\\
x_1 + \lambda\\
x_1 + x_2 - 1
\end{array}\right]}

    \Rightarrow{ \lambda = -\dfrac{1}{2}}, 
               { x_1 = \dfrac{1}{2}}, 
               { x_2 = \dfrac{1}{2}}


	
	
	\item [(c)]
	
	Solution
\end{enumerate}

%-----------------------------------------------------------------------------------------------

\section*{Exercise 4: Lagrange Multipliers and Bordered Hessian}
\begin{enumerate}
	\item [(a)] 
	\item [(b)] 
	\item [(c)] 
\end{enumerate}

%-----------------------------------------------------------------------------------------------

\section*{Exercise 5: An Application to Economics}
\begin{itemize}
	\item [(a)] 
	\item [(b)] 
	\item [(c)]
\end{itemize}

%-----------------------------------------------------------------------------------------------

\section*{Exercise 6: Maximisation Using the KKT Conditions}

%-----------------------------------------------------------------------------------------------



\end{document}
